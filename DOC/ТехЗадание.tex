\section{Техническое задание}
\subsection{Основание для разработки}

Основанием для разработки является задание на выпускную квалификационную работу бакалавра "<Разработка корпоративного мессенджера с системой групповых чатов и end-to-end шифрованием">.

\subsection{Цель и назначение разработки}

Основной задачей выпускной квалификационной работы является создание безопасного мессенджера для внутренней коммуникации сотрудников компании с поддержкой группового взаимодействия.

Цели разработки:
\begin{itemize}
	\item Реализация защищенного канала обмена сообщениями
	\item Создание системы управления групповыми чатами
	\item Обеспечение кросс-платформенной совместимости
	\item Интеграция с корпоративной системой аутентификации
\end{itemize}

Задачами разработки являются:
\begin{itemize}
	\item Создание модуля шифрования сообщений
	\item Реализация интерфейса чатов с историей переписки
	\item Разработка системы ролевого доступа для групп
	\item Интеграция API для работы с файлами
	\item Реализация поиска по истории сообщений
\end{itemize}

\subsection{Требования пользователя к интерфейсу}

Мессенджер должен включать:
\begin{itemize}
	\item Панель списка чатов и групп
	\item Окно активного чата с историей сообщений
	\item Систему уведомлений
	\item Контекстное меню управления группами
	\item Панель быстрого поиска
\end{itemize}

Композиция интерфейса представлена на рисунке ~\ref{fig:interface}.

\begin{figure}[ht]
	\includegraphics[width=1\linewidth]{interface}
	\caption{Схема интерфейса мессенджера}
	\label{fig:interface}
\end{figure}

\subsection{Моделирование вариантов использования}

Диаграмма вариантов использования построена с использованием UML и включает следующие основные сценарии:
\begin{enumerate}
	\item Авторизация пользователя
	\item Создание группового чата
	\item Отправка текстовых и файловых сообщений
	\item Управление участниками группы
	\item Поиск в истории переписки
	\item Настройка параметров безопасности
\end{enumerate}

\subsection{Требования к оформлению документации}

Разработка программной документации и программного изделия должна производиться согласно ГОСТ 19.102-77 и ГОСТ 34.601-90. Единая система программной документации.