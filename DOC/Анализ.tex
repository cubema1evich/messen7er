\section{Анализ предметной области}
\subsection{Современные тенденции корпоративных коммуникаций}

Корпоративные мессенджеры - это специализированные платформы, ориентированные на оптимизацию рабочих процессов. В отличие от массовых сервисов (WhatsApp, Telegram), они предоставляют инструменты для структурированной коммуникации внутри организаций, интеграции с бизнес-приложениями и контроля данных.
Основными функциями корпоративного мессенджера являются:
\begin{itemize}
	\item Обмен сообщениями и файлами
	\item Безопасность и контроль данных
	\item Организация коммуникации
	\item Групповые чаты сотрудников (общие, тематические, отделов, команд)
	\item Личные чаты между сотрудниками
	\item Интеграция с другими сервисами и системами для управления задачами
	\item Управление доступом для пользователей (администратор, модератор, сотрудник, клиент, иной пользователь)
	\item Кроссплатформенность
	\item Встроенная IP-телефония
	\item Таймер автоматического удаления сообщений и файлов
\end{itemize}

На отечественном, так и на зарубежном рынке были успешно представлены такие решения:
\begin{itemize}
	\item Microsoft Teams — глубокая интеграция с экосистемой Microsoft, поддержка многотысячных команд.
	\item Slack — гибкие настройки рабочих процессов через API и приложения (например, интеграция с GitHub).
	\item Zulip — уникальная система потоков в чатах, упрощающая отслеживание тем.
	\item СберЧаты (СберБизнес) — поддержка E2E-шифрования, интеграция с сервисами Сбера.
	\item eXpress  — функциональность классического мессенджера с возможностями для защищенной корпоративной коммуникации и общения команд. 
\end{itemize}

Корпоративные чаты, в отличие от обычных мессенджеров, ориентированы на рабочие процессы. Вы общаетесь преимущественно с коллегами, а если нужно пригласить к беседе клиента или подрядчика, то выдаете ему гостевой доступ или временную учетную запись. Такие корпоративные чаты позволяют лучше организовать процессы.

Рынок корпоративных мессенджеров предлагает различные модели установки — на своих серверах или в облаке компании. Это позволяет контролировать безопасность самостоятельно, выбирая вариант, который подходит больше всего.

Ситуация с зарубежными мессенджерами в России в 2025 году

Slack и MS Teams ушли с российского рынка, а западные мессенджеры блокируются.

Корпоративные мессенджеры 2025 года должны удовлетворять ключевым требованиям бизнеса:
\begin{itemize}
	\item \textbf{Безопасность}: Шифрование данных, двухфакторная аутентификация, контроль доступа
	\item \textbf{Гибкость}: Поддержка групповых и личных чатов, системы уведомлений, интеграция с корпоративными сервисами
	\item \textbf{Кроссплатформенность}: Веб-интерфейс + мобильные приложения
	\item \textbf{Производительность}: Минимальные задержки при обмене сообщениями
\end{itemize}

{Ключевые инновации проекта}
Гибкая система управления группами с ролевой моделью:
\begin{itemize}
	\item Администратор
	\item Модератор
	\item Сотрудник
	\item Клиент (заказчик, подрядчик)
\end{itemize}
		

