\section{Анализ предметной области}
\subsection{Современные тенденции корпоративных коммуникаций}

Корпоративные мессенджеры или корпоративные чаты — это сервисы для звонков и обмена сообщениями между сотрудниками компании, стартапа или бизнеса. Такие мессенджеры позволяют наладить коммуникацию между отделами и командами, надежно защитить передаваемые файлы и документы и в некоторых случаях даже работать без выхода в интернет.  

Корпоративные чаты, в отличие от обычных мессенджеров, ориентированы на рабочие процессы. Вы общаетесь преимущественно с коллегами, а если нужно пригласить к беседе клиента или подрядчика, то выдаете ему гостевой доступ или временную учетную запись. 

Также корпоративные чаты позволяют лучше организовать процессы. Например, в некоторых мессенджерах есть встроенный календарь для назначения встреч и менеджер задач. Встроенная IP-телефония позволяет совершать звонки из мессенджера напрямую на телефон, а интеграции с онлайн-документами — совместно работать над файлами.

Рынок корпоративных мессенджеров предлагает различные модели установки — на своих серверах или в облаке компании. Это позволяет контролировать безопасность самостоятельно, выбирая вариант, который подходит больше всего.

Ситуация с зарубежными мессенджерами в России в 2025 году

Slack и MS Teams ушли с российского рынка, а западные мессенджеры блокируются. Осенью 2024 года был заблокирован Discord, а в декабре — Viber.

Следующим может оказаться и WhatsApp, который в декабре внесли в реестр организаторов распространения информации. В Государственной Думе РФ уже заявили, что сервис могут заблокировать, но пока точного подтверждения этому нет.

Несмотря на то, что WhatsApp пользуются многие люди, в том числе и в рабочих целях, есть смысл присмотреться к альтернативам — корпоративным мессенджерам, которые остаются доступными в России в 2025 году или разработаны в нашей стране и им не грозит риск блокировки.зация центров аддитивного производства. Сегодня аддитивные технологии внедряются в самые сложные и наукоемкие отрасли: атомную промышленность, аэрокосмическую индустрию, медицину, автомобилестроение и многие другие. Применение АТ решает задачи по снижению стоимости, сокращения срока изготовления изделий и обеспечение высокой персонализации деталей.31
\section{Анализ предметной области}
\subsection{Современные требования к корпоративным мессенджерам}

Корпоративные мессенджеры 2025 года должны удовлетворять ключевым требованиям бизнеса:
\begin{itemize}
	\item \textbf{Безопасность}: Шифрование данных, двухфакторная аутентификация, контроль доступа
	\item \textbf{Гибкость}: Поддержка групповых и личных чатов, системы уведомлений, интеграция с корпоративными сервисами
	\item \textbf{Кроссплатформенность}: Веб-интерфейс + мобильные приложения
	\item \textbf{Производительность}: Минимальные задержки при обмене сообщениями
\end{itemize}

\subsection{Технологический стек реализации}
Разработанное решение использует следующие технологии:

\renewcommand{\arraystretch}{0.8}
\begin{xltabular}{\textwidth}{|X|p{2.5cm}|>{\setlength{\baselineskip}{0.7\baselineskip}}p{4.85cm}|>{\setlength{\baselineskip}{0.7\baselineskip}}p{4.85cm}|}
	\caption{Технологический стек реализации\label{tab:stack}}\\
	\hline 
	\centrow{\setlength{\baselineskip}{0.7\baselineskip} Компонент} & 
	\centrow{\setlength{\baselineskip}{0.7\baselineskip} Технологии} & 
	\centrow{Описание} & 
	\centrow{Ключевые особенности} \\
	\hline 
	\endfirsthead
	
	\caption*{Продолжение таблицы \ref{tab:stack}}\\
	\hline 
	\centrow{1} & \centrow{2} & \centrow{3} & \centrow{4}\\ 
	\hline
	\endhead
	
	Фронтенд & Vanilla JS, HTML5, CSS3 & Клиентская часть приложения & 
	\begin{itemize}
		\setlength\itemsep{0em}
		\item Нативный JavaScript
		\item Адаптивная верстка
		\item Кастомные элементы UI
	\end{itemize} \\
	\hline
	
	Бэкенд & Python (WSGI), REST API & Серверная логика & 
	\begin{itemize}
		\setlength\itemsep{0em}
		\item Минимальные зависимости
		\item Модульная архитектура
		\item Асинхронная обработка
	\end{itemize} \\
	\hline
	
	База данных & SQLite с ORM & Хранение данных & 
	\begin{itemize}
		\setlength\itemsep{0em}
		\item Локальное хранение
		\item Транзакции ACID
		\item Миграции схемы
	\end{itemize} \\
	\hline
\end{xltabular}
\renewcommand{\arraystretch}{1.0}

\subsection{Архитектурные особенности}

Ключевые особенности:
\begin{itemize}
	\item Трехслойная архитектура (клиент-сервер-БД)
	\item Микросервисная организация API
	\item Шаблон MVC для веб-интерфейса
	\item Асинхронная обработка сообщений
\end{itemize}

\subsection{Сравнение с существующими решениями}

\renewcommand{\arraystretch}{0.8}
\begin{xltabular}{\textwidth}{|X|p{2.5cm}|p{2.5cm}|p{2.5cm}|}
	\caption{Сравнительный анализ мессенджеров\label{tab:compare}}\\
	\hline 
	\centrow{\setlength{\baselineskip}{0.7\baselineskip} Характеристика} & 
	\centrow{\setlength{\baselineskip}{0.7\baselineskip} Наше решение} & 
	\centrow{Slack} & 
	\centrow{Microsoft Teams} \\
	\hline 
	\endfirsthead
	
	\caption*{Продолжение таблицы \ref{tab:compare}}\\
	\hline 
	\centrow{1} & \centrow{2} & \centrow{3} & \centrow{4}\\ 
	\hline
	\endhead
	
	Локальное хранение данных & + & - & - \\ \hline
	Шифрование сообщений & AES-128 & AES-256 & AES-256 \\ \hline
	Поддержка групповых чатов & + & + & + \\ \hline
	Личные сообщения & + & + & + \\ \hline
	API для интеграций & REST & REST + Websockets & GraphQL \\ \hline
	Мобильные приложения & + & + & + \\ \hline
	Стоимость лицензии & Бесплатно & От \$6.67/польз. & От \$4.00/польз. \\ \hline
\end{xltabular}
\renewcommand{\arraystretch}{1.0}

\newpage

\subsection{Ключевые инновации проекта}

Гибкая система управления группами с ролевой моделью:
\begin{itemize}
	\item Владелец группы
	\item Администраторы
	\item Обычные участники
\end{itemize}
		
Гибридная система обновлений:
\begin{itemize}
	\item Long-polling для групповых чатов
	\item WebSocket для личных сообщений
\end{itemize}
	
Система динамического рендеринга сообщений:
\begin{itemize}
	\item Виртуализация DOM
	\item Ленивая загрузка истории
	\item Кеширование на клиенте
\end{itemize}

